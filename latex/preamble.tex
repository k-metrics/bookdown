% --- 参考資料 ----------------------------------------------------------------
% https://github.com/Gedevan-Aleksizde/Japan.R2019/blob/master/latex/preamble.tex
% https://teastat.blogspot.com/2019/01/bookdown.html


% --- Packages ----------------------------------------------------------------
% 日本語とtufte, kableExtraを使うために必要なTeXパッケージ指定
%  A4 210mm x 297mm
%   \usepackage[a4paper, total={6.5in, 9.5in}]{geometry}
%   \usepackage{indentfirst}   # tinytexのリポジトリには存在しない?

\usepackage[a4paper, total={160mm, 247mm}, left=25mm, top=25mm]{geometry}
% \usepackage[pdfbox,tombo]{gentombow}  % トンボを設定する場合は有効にする
\usepackage{ifthen}                     % 条件分岐用 \ifthenelse{条件}{T}{F}
\usepackage{booktabs}                   % ここからkableExtra用パッケージ
\usepackage{longtable}                  % 
\usepackage{array}                      % 
\usepackage{multirow}                   % 
\usepackage{wrapfig}                    % 
\usepackage{float}                      % 
\usepackage{colortbl}                   % 
\usepackage{pdflscape}                  % 
\usepackage{tabu}                       % 
\usepackage{threeparttable}             % 
\usepackage{threeparttablex}            % 
\usepackage[normalem]{ulem}             % 
\usepackage{inputenc}                   % 
\usepackage{makecell}                   % 
\usepackage{xcolor}                     % ここまでkableExtra用
\usepackage{amsmath}                    % 
\usepackage{fontawesome5}               % fontawesomeを使うために必要
\usepackage{subfig}
\usepackage{xeCJK}                      % 以下、日本語フォント用に必要
\usepackage[noto]{zxjafont}             % Linux環境ではこちを指定
% \usepackage[haranoaji]{zxjafont}      % Windows環境ではこちらを指定する
\usepackage{zxjatype}
\usepackage{pxrubrica}                  % ルビ用

% --- Fonts -------------------------------------------------------------------
% フォントしては index.html でも可能(pandoc用オプションは index.htmlにて)
\setmonofont{Source Han Code JP}
\setjamonofont{Source Han Code JP}

% ## 日本語フォントの扱いについてはzxjafontパッケージの解説を参照のこと
% # https://mirror.las.iastate.edu/tex-archive/language/japanese/zxjafont/zxjafont.pdf
% #
% ## Windows環境ではなぜかNotoフォントが認識されないので源ノシリーズベースの
% ## 原ノ味フォントかIPAexフォントを利用する(原ノ味はtlmgrでインストール可)
% # \usepackage[haranoaji]{zxjafont}
% # \usepackage[ipaex]{zxjafont}
% #
% ## Windows環境でNotoフォントを指定したい場合は以下のようにheader-includeで
% ## 個別に指定する(setCJKxxxfotnの指定は必要?)
% # \setmainfont{NotoSerifCJKjp-Regular.otf}[BoldFont=NotoSerifCJKjp-Bold.otf]
% # \setsansfont{NotoSansCJKjp-Regular.otf}[BoldFont=NotoSansCJKjp-Bold.otf]
% # \setmonofont{NotoSansMonoCJKjp-Regular.otf}[BoldFont=NotoSansMonoCJKjp-Bold.otf]
% ## モノフォントは源ノ角コード(Source Code Proの日本語版)がおすゝめ
% # \setmonofont{SourceHanCodeJP-Regular.otf}[BoldFont=SourceHanCodeJPS-Bold.otf]
% ## モノフォントのみ入れ替えフォントのスケーリングを調整する
% \mainfontoptions: "Scale=MatchUppercase"
% \setmonofontoptions{Scale=0.8}      % 0.96がベストらしいが…
% \setCJKoptions{Scale=1.0"}
% ## Windows環境でモノフォントを変更するにはheader-inludeで以下の指定を行う
% # \setmonofont{SourceCodePro-Regular.ttf}[BoldFont=SourceCodePro-Bold.ttf]


% --- 空ページにヘッダーを設定する(機能していない) --------------------------
% \makeatletter
% \def\emptypage@emptypage{
%     \hbox{}
%     \thispagestyle{headings}
%     \newpage
% }
% \def\cleardoublepage{
%         \clearpage
%         \if@twoside
%             \ifodd\c@page
%                 % do nothing
%             \else
%                 \emptypage@emptypage
%             \fi
%         \fi
%     }
% \makeatother


% --- 参考文献一覧を目次に表示させる ------------------------------------------
% \makeatletter
% \renewenvironment{thebibliography}[1]
% {\chapter*{\bibname\@mkboth{\bibname}{\bibname}}
%    \addcontentsline{toc}{chapter}{\bibname}%
%    \list{\@biblabel{\@arabic\c@enumiv}}%
%         {\settowidth\labelwidth{\@biblabel{#1}}%
%          \leftmargin\labelwidth
%          \advance\leftmargin\labelsep
%          \@openbib@code
%          \usecounter{enumiv}%
%          \let\p@enumiv\@empty
%          \renewcommand\theenumiv{\@arabic\c@enumiv}}%
%    \sloppy
%    \clubpenalty4000
%    \@clubpenalty\clubpenalty
%    \widowpenalty4000%
%    \sfcode`\.\@m}
%   {\def\@noitemerr
%     {\@latex@warning{Empty `thebibliography' environment}}%
%    \endlist}
% \makeatother

