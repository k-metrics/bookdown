% --- 参考資料 ----------------------------------------------------------------
% https://github.com/Gedevan-Aleksizde/Japan.R2019/blob/master/latex/preamble.tex
% https://teastat.blogspot.com/2019/01/bookdown.html

% --- 未解決事項 --------------------------------------------------------------
% 1)knitr時に以下のワーニングが出る
%  警告メッセージ: 
% Package xeCJK Warning: Redefining CJKfamily `\CJKttdefault'
% (xeCJK)                (NotoSansCJKjp-Regular). 
% 
% 2) Bibliographyページを目次に表示できない
% 
% 3) 文頭一文字インデントを有効にできない
%    \usepackage{indentfirst}   # インストールできない
% 

% --- Packages ----------------------------------------------------------------
% 日本語とtufte, kableExtraを使うために必要なTeXパッケージ指定
%  A4 210mm x 297mm
%   \usepackage[a4paper, total={6.5in, 9.5in}]{geometry}
%   \usepackage{indentfirst}   # tinytexのリポジトリには存在しない?

\usepackage[a4paper, total={160mm, 247mm}, left=25mm, top=25mm]{geometry}
% \usepackage[pdfbox,tombo]{gentombow}  % トンボを設定する場合は有効にする
\usepackage{ifthen}                     % 条件分岐用 \ifthenelse{条件}{T}{F}
\usepackage{booktabs}                   % ここからkableExtra用パッケージ
\usepackage{longtable}                  % 
\usepackage{array}                      % 
\usepackage{multirow}                   % 
\usepackage{wrapfig}                    % 
\usepackage{float}                      % 
\usepackage{colortbl}                   % 
\usepackage{pdflscape}                  % 
\usepackage{tabu}                       % 
\usepackage{threeparttable}             % 
\usepackage{threeparttablex}            % 
\usepackage[normalem]{ulem}             % 
\usepackage{inputenc}                   % 
\usepackage{makecell}                   % 
\usepackage{xcolor}                     % ここまでkableExtra用
\usepackage{amsmath}                    % 
\usepackage{fontawesome5}               % fontawesomeを使うために必要
\usepackage{subfig}
\usepackage{xeCJK}                      % 以下、日本語フォント用に必要
\usepackage[noto]{zxjafont}             % Linux環境ではこちを指定
% \usepackage[haranoaji]{zxjafont}      % Windows環境ではこちらを指定する
\usepackage{zxjatype}
\usepackage{pxrubrica}                  % ルビ用
\usepackage{hyperref}                   % ハイパーリンク用必要?
\usepackage{tcolorbox}                  % 色付き囲い

% --- Index ------------------------------------------------------------------
% https://texwiki.texjp.org/?%E7%B4%A2%E5%BC%95%E4%BD%9C%E6%88%90
% これを指定するとIndex(索引)は作成されるが参照ページがズレる
% 中間ファイルの.indではページはズレていないので、その後の結合処理がおかしい
\usepackage{makeidx}
\makeindex
% \usepackage{showidx}                  % 索引確認用

% --- Table of Contentes ------------------------------------------------------
% TOCにLOT(List of Tables), LOF(List of Figures), Bibliography, Indexを表示
\usepackage[nottoc]{tocbibind}

% --- Fonts -------------------------------------------------------------------
% フォントしては index.html でも可能(pandoc用オプションは index.htmlにて)
% \setCJKmonofont{Source Han Code JP}
\setmonofont{Source Han Code JP}
% \setjamonofont{Source Han Code JP}

% ## 日本語フォントの扱いについてはzxjafontパッケージの解説を参照のこと
% # https://mirror.las.iastate.edu/tex-archive/language/japanese/zxjafont/zxjafont.pdf
% #
% ## Windows環境ではなぜかNotoフォントが認識されないので源ノシリーズベースの
% ## 原ノ味フォントかIPAexフォントを利用する(原ノ味はtlmgrでインストール可)
% # \usepackage[haranoaji]{zxjafont}
% # \usepackage[ipaex]{zxjafont}
% #
% ## Windows環境でNotoフォントを指定したい場合は以下のようにheader-includeで
% ## 個別に指定する(setCJKxxxfotnの指定は必要?)
% # \setmainfont{NotoSerifCJKjp-Regular.otf}[BoldFont=NotoSerifCJKjp-Bold.otf]
% # \setsansfont{NotoSansCJKjp-Regular.otf}[BoldFont=NotoSansCJKjp-Bold.otf]
% # \setmonofont{NotoSansMonoCJKjp-Regular.otf}[BoldFont=NotoSansMonoCJKjp-Bold.otf]
% ## モノフォントは源ノ角コード(Source Code Proの日本語版)がおすゝめ
% # \setmonofont{SourceHanCodeJP-Regular.otf}[BoldFont=SourceHanCodeJPS-Bold.otf]

% --- Fonts -------------------------------------------------------------------
% https://teastat.blogspot.com/2019/01/bookdown.html
% 図の位置を固定する
\floatplacement{figure}{H}    % 図位置を固定する
\floatplacement{table}{H}     % 表位置を固定する

% 引数の意味
%   H:絶対に指定箇所に置く(float.sty の効果)
%   h:できれば指定箇所に置く。少しでも無理ならあきらめて次の候補へ
%   t:版面上端に置く
%   b:版面下端に置く
%   p:単独のページに置く

% --- 色付き囲いの定義 --------------------------------------------------------
% https://marukunalufd0123.hatenablog.com/entry/2019/03/15/071717
% 利用する場合は先頭に全角空白を入れること
% Round Box
\newtcolorbox{info-box}[2][]{colback=cyan!5!white, colframe=cyan!60!black,
title={#2}, #1}
\newtcolorbox{warning-box}[2][]{colback=orange!5!white, 
colframe=orange!80!black, title={#2}, #1}
\newtcolorbox{error-box}[2][]{colback=red!5!white, colframe=red!75!black,
title={#2}, #1}

% Square Box
% \newtcolorbox{info-box}[2][]{colback=cyan!5!white, arc=0pt, outer arc=0pt,
% colframe=cyan!60!black, title={#2}, #1}
% \newtcolorbox{warning-box}[2][]{colback=orange!5!white, arc=0pt, outer arc=0pt,
% colframe=orange!80!black, title={#2}, #1}
% \newtcolorbox{error-box}[2][]{colback=red!5!white, arc=0pt, outer arc=0pt,
% colframe=red!75!black, title={#2}, #1}

